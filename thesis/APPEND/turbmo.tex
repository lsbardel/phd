%
%
%
%
\chapter{Spalart Allmaras Turbulence Model}
\label{turbulence.chap}
\headc{Spalart Allmaras Turbulence Model}
\setcounter{footnote}{0}
%
 Turbulence modeling, together with grid generation and algorithm development,
 is one of the three key elements in CFD. However, while for grid generation and
 algorithm development very precise mathematical theories have evolved, mathematical
 models that approximate the physical behaviour of turbulent flows achieved far
 less precision. The reason for that is because turbulence
 modelling try to approximate an extremely complicated phenomena.
 The turbulence model used throughout the research project belong to
 class of models in where the the turbulence length scale is related to
 some typical flow dimension.
%
%
%
%
\section{One Equation Spalart-Allmaras Model}
%
 Using the same convection as for the mean flow equations,
 the Spalart-Allmaras \citeyear{Spalart:1} one equation turbulence model can be
 written in a ALE integral conservative form as

%
\beq
  \fpdt{} \int_{{\cal V}\left(t\right)} \rho\upsilon\sm{t} d{\cal V} +
  \oint_{{\cal S}\left(t\right)} \left[ 
  \vec{F}\sm{t} - \frac{1}{Re}\vec{G}\sm{t}\right]
  \cdot\vec{n}\, d{\cal S} =
  \int_{{\cal V}\left(t\right)} S\sm{t}\,d{\cal V}
  \label{turbulent_1}
\eeq
%
 where the flux vector $\vec{F\sm{t}}$ and $\vec{G\sm{t}}$
 and the source term $S\sm{t}$ are given by

%
\beq
  \vec{F}\sm{t} &=& \rho\upsilon\sm{t}\,\vec{u}
  \label{turbulent_2}\\
  \vec{G}\sm{t} &=& \frac{1}{\sigma}\left(\mu\sm{l} + \rho\upsilon\sm{t}\right)\nabl \upsilon\sm{t}
  \label{turbulent_3}\\
  S\sm{t} &=& c\sm{b1}P\,\rho\upsilon\sm{t}
  + \frac{\rho}{Re}\left[\frac{c\sm{b2}}{\sigma}\left(\nabl\upsilon\sm{t}\right)\se{2}
  - c\sm{w1}f\sm{w}\left(\frac{\upsilon\sm{t}}{d}\right)\se{2}\right]
  \label{turbulent_4}
\eeq
%  - \frac{1}{\rho}\left(c\sm{w1}f\sm{w}\left(\frac{\rho\upsilon\sm{t}}{d}\right)\se{2}\right]
%  + \frac{\mu\sm{l} + \rho\upsilon\sm{t}}{\sigma}\nabl \upsilon\sm{t}\cdot\nabl\rho\right)\right]
%
 The turbulent eddy viscosity $\mu\sm{t}$ is related to the
 turbulent unknown $\rho\upsilon\sm{t}$ through the relation

%
\beq
  \mu\sm{t} = \rho\upsilon\sm{t}\,f\sm{v1}
  \label{turbulent_5}
\eeq
%
 in equations (\ref{turbulent_1}-\ref{turbulent_5}) the following functions are used

%
\beq
  f\sm{v1} &=& \frac{\chi\se{3}}{\chi\sm{3} + c\sm{v1}\se{3}}\\
  \chi &=& \frac{\rho\upsilon\sm{t}}{\mu\sm{l}}\\
  P &=& \left|\nabl \times \vec{u}\right| + \xi f\sm{v2}\\
  \xi &=& \frac{1}{Re}\frac{\upsilon\sm{t}}{\kappa\se{2} d\se{2}}\\
  f\sm{v2} &=& 1 - \frac{\chi}{1+\chi f\sm{v1}}\\
  f\sm{w} &=& g\left(\frac{1+c\sm{w3}\se{6}}{g\se{6}+c\sm{w3}\se{6}}\right)\se{1/6}\\
  g &=& r + c\sm{w2}\left(r\se{6} - r\right)\\
  r &=& \frac{\xi}{P}
\eeq
%
 For all calculations the following constants are used

%
\beq
  \begin{array}{rclrclrcl}
      \kappa  &=& 0.41, &
      \sigma  &=& \frac{2}{3}, &
     c\sm{b1} &=& 0.1355,\\
    c\sm{b2}  &=& 0.622, &
    c\sm{w1}  &=& \frac{c\sm{b1}}{\kappa\se{2}} + \frac{1+c\sm{b2}}{\sigma}, &
    c\sm{w2}  &=& 0.3,\\
    c\sm{w3}  &=& 2.0, &
    c\sm{v1}  &=& 7.1
  \end{array}
\eeq
%
%
%
%
\section{Linearised Turbulence Model}
%
 Following the same procedure as for the mean flow equations, the Spalart-Allmaras turbulence
 model can be written in a frequency domain linearised form as

%
\beq
 \fpd{}{\tau}\int\sm{\overline{\cal V}} \widehat{\rho\upsilon}\sm{t} d\overline{\cal V} +
 \oint_{\overline{\cal S}}\left(\widehat{\vec{F}}\sm{t}-
       \frac{1}{Re}\widehat{\vec{G}}\sm{t}\right)
 \cdot \overline{\vec{n} d{\cal S}}  &-&
 \int_{\overline{\cal V}} \left(i\omega \widehat{\rho\upsilon}\sm{t} + \widehat{S}\sm{t}\right)
 d\overline{\cal V} =
 \nonumber\\
 &&
 \widehat{H}\sm{t}\left(\overline{\rho\upsilon}\sm{t},
  \overline{\bf U}, \widehat{\vec{x}}\right)
 \label{linear_turbulent_1}
\eeq
%
 The linearised inviscid and viscous fluxes are given by the two relations

%
\beq
 \widehat{\vec{F}}\sm{t} &=& \widehat{\rho\upsilon}\sm{t}\left(\overline{\vec{v}}-
                             \vec{\Omega}\times\overline{\vec{x}}\right) +
                             \overline{\rho\upsilon}\sm{t}\widehat{\vec{v}}
 \label{linear_turbulent_2}\\
 \widehat{\vec{G}}\sm{t} &=& \left(\overline{\mu}\sm{l}+\overline{\rho}
                             \overline{\upsilon}\sm{t}\right) \nabl \widehat{\upsilon}\sm{t}+
                             \left(\widehat{\mu}\sm{l}+\widehat{\rho\upsilon}\sm{t}\right)
                             \nabl\overline{\upsilon}\sm{t}
 \label{linear_turbulent_3}
\eeq
%
 The linearised eddy viscosity is given by

%
\beq
  \widehat{\mu\sm{t}} = \widehat{\rho\upsilon}\sm{t}\,f\sm{v1}
  \label{turbulent_linear.eq}
\eeq
%

 The terms $\widehat{H}$ and the right-hand side of equation (\ref{linear_turbulent_1})
 indicates the summation of inhomogeneous terms arising from the grid motion.

%
\beq
 \widehat{H}\sm{t} &=& \int_{\overline{\cal V}} \left(i\omega\overline{\rho\upsilon}\sm{t}+
                           \overline{S}\sm{t}\right)d\widehat{\cal V} - 
                       \oint_{\overline{\cal S}}\left(\overline{\vec{F}}\sm{t}-
                       \frac{1}{Re}\overline{\vec{G}}\sm{t}\right) \cdot
                       \widehat{{\vec{n}} d{\cal S}} 
  \nonumber\\
 &+& \oint_{\overline{\cal S}}
  \overline{\rho\upsilon}\sm{t}\left(\vec{\Omega}\times\widehat{\vec{x}}
                    -i\omega\widehat{\vec{x}}\right)
  \cdot \overline{{\vec{n}} d{\cal S}}
 \label{linear_turbulent_4}
\eeq
%
The linearised source term $\widehat{S}\sm{t}$ is given by the relation

%
\beq
  \widehat{S}\sm{t} =
  c\sm{b1}\left(
  \overline{P}\,\widehat{\rho\upsilon}\sm{t} +
  \widehat{P}\,\overline{\rho\upsilon}\sm{t}\right)
  +\frac{2\overline{\rho}}{Re}\left[\frac{c\sm{b2}}{\sigma}
  \left(\nabl\overline{\upsilon}\sm{t}\right)\left(\nabl\widehat{\upsilon}\sm{t}\right)
  - c\sm{w1}f\sm{w}
 \frac{\overline{\upsilon}\sm{t}\widehat{\upsilon}\sm{t}}{d\se{2}}\right]
 \label{s_linear.eq}
\eeq
%
 with

%
\beq
  \widehat{P} =  \left|\nabl \times \widehat{\vec{u}}\right| +
  \frac{f\sm{v2}}{Re}\frac{\widehat{\upsilon}\sm{t}}{\kappa\se{2} d\se{2}}
 \label{p_linear.eq}
\eeq
%
 The functions $f\sm{v1}$, $f\sm{v2}$ and $f\sm{w}$ in
 (\ref{turbulent_linear.eq}), (\ref{s_linear.eq}) and
 (\ref{p_linear.eq}) are evaluated using the steady-state unknowns.
