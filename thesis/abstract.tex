%
%
%
%
%
\chapter*{Preface}
\addcontentsline{toc}{chapter}{\hspace{5mm} Preface}
%
 The flow through the various stages of turbomachines is inherently unsteady,
 primarily due to the relative motion between adjacent stator and rotor
 blade rows, though other factors such as aerodynamic mistuning, temperature
 distortions due to blocked burners may also become significant. The
 unsteadiness can persist through several stages and may excite either a
 structural or an acoustic mode and such resonances can cause  severe
 vibration problems.  The overall picture is further complicated because of
 other events such as surge, stall and rotating stall. In any case, such
 unstable phenomena are increasingly becoming limiting factors in developing
 improved-efficiency designs, especially when flexible,
 slender and unshrouded blades are used.
 From an industrial prospective, it is clear that there is a
 pressing need for formulating validated predictive models to reduce
 the design cycle and, consequently, cost.

 Such is the aim of the present
 work and two different methods of simulating 3D viscous unsteady
 turbomachinery flows were developed: non-linear time marching and
 frequency linearised method. The contributions to CFD modelling include an
 efficient discretisation of the viscous terms, the linearisation of the
 turbulence model and the development of a multigrid methodology for viscous
 unsteady flows. Although both the non-linear and linearised methods can use
 general unstructured meshes, a novel semi-structured mesh generator was
 developed  for computational efficiency. Over standard structured meshes,
 such a route provides a significant improvement
 both in the smoothness of the grid-spacing and in resolving
 particular aspects of the blade passage geometry. Leading-
 and trailing-edge regions are discretised without generating
 superfluous points, while wakes and shocks can be
 captured using local refinement techniques.
 A particular feature of the methodology is the use of mixed element grids,
 consisting of triangles
 and quadrilaterals in 2D, and of tetrahedra, pyramids, triangular
 prisms and hexahedra in 3D.

 The final part of the thesis presents a detailed unsteady flow analysis
 for forced response prediction of a 3D transonic turbine stage.
 The main aim is to compare the performance of the non-linear
 time-marching and linearised flow simulation methods when predicting
 the unsteadiness due to stator-rotor interactions.
 The results obtained with the two method were found to be quantitatively
 similar although non-linearities are evident in the leading-edge
 region where the potential flow interaction is dominant.

 Finally, an analysis of the different unsteady effects caused by potential flow
 and wake-blade interactions was also made.
