%
%
%
%
%
\section{Objectives}
\headb{Introduction}{Objectives}
\label{objective.sec}
%
 This thesis deals with the development and application of a numerical
 tool for the simulation of unsteady turbomachinery flows for forced response
 predictions. The main goal is not only to produce a method suitable
 for industrial design but also to explore its bounds of applicability to
 practical cases.
 Both linear and non-linear time-marching unsteady flow models 
 have been implemented using an advanced finite volume algorithm on
 unstructured meshes.
 The research code, called ALiNNS (Advanced Linear-Nonlinear Navier-Stokes code),
 uses state-of-the-art CFD techniques for both spatial discretisation and temporal
 integration of the governing flow equations. The main features
 of the methodology are the use of mixed-element grids (also called hybrid grids)
 and a preconditioned agglomeration multigrid.
 Moreover a semi-unstructured mesh generator, LEVMAP (LEVel MAPping), was developed in order
 to exploit the optimum discretisation features of such an approach on 3D
 turbomachinery passages.
 LEVMAP and ALiNNS were used to predict the unsteady pressure fluctuations,
 due to relative blade rotation, on a typical HP turbine stage.
 The analysis was carried out using both linearised and non-linear aerodynamics.
 Comparisons with experimental data were made
 in order to assess the validity and the range of applicability of the linear
 method.

 The main objectives of the research project can be summarised as follow:
%
\begin{itemize}
%
\item
 The development of a semi-structured mesh generator for an efficient representation
 of 3D turbomachinery blades.
%
\item
 The development of an efficient viscous flux discretisation algorithm for
 mixed element grids and of a preconditioned agglomeration multigrid for
 efficient time integration.
%
\item
 The development of a 3D linearised viscous unsteady flow model.
%
\item
 The application of the developed non-linear and linearised flow models
 to a representative case in order to asses the applicability bounds
 of the linearised methodology.
%
\end{itemize}
%
%
\section{Contributions of the Thesis}
\headb{Introduction}{Contributions of the thesis}
\label{contributions.sec}
%
\subsection{CFD methods}
 The main contributions are summarised below.
%
\paragraph{Semi-structured meshes.}
%
 The discretisation methodology uses a combination of structured and unstructured meshes,
 the former in the radial direction and the latter in the axial and tangential
 directions in order to exploit the fact that blade-like structures
 are not strongly 3D since the radial variation
 is usually small. Such a formulation was found to
 have a number of advantages over its structured counterparts.
 There is a significant improvement
 in the smoothness of the grid-spacing and in capturing
 particular aspects of the blade passage geometry. It was also found that
 the leading- and trailing-edge regions could be  discretised without generating
 superfluous points in the far field and that further refinements of the mesh  
 to capture wake and shock effects were relatively easy to implement.
 The methodology is reported by Sbardella at al. \citeyear{Luca:3,Luca:9}
 and Sbardella \citeyear{Luca:5}. A detailed description is given in
 Chapter \ref{mesh.chap}.
%
\paragraph{Hybrid-grid flow-solver.}
%
 Compared with their structured counterparts, standard unstructured grid
 solvers have lower computational efficiency in terms of speed and storage.
 Unstructured grids often use tetrahedral elements only,
 an approach which often leads to numerical problems when the region to be discretised
 has a preferred direction such as the boundary layer for a
 high-Reynolds number flow. However, there are no fundamental difficulties in 
 extending tetrahedral meshes to include further element types such as triangular prisms,
 pentahedra and hexahedra.
 Although both the discretisation of the computational domain and the flow solver
 will become more complex,  such a mixed-element approach will offer
 a better, more efficient approximation than using tetrahedral elements only.
 For instance, hexahedral elements will handle boundary layer flows much better
 than tetrahedral elements because they can be made very slender without
 creating excessively small or large internal angles. 
 In order to handle mixed-element meshes, the spatial discretisation
 of the governing equations needs to be formulated in such a way
 that the numerical algorithms can be applied in a uniform way to all
 element types.
 Chapter \ref{flow_model.chap} describes the development and application
 of a finite volume scheme for the solution of the
 Favre-averaged Navier-Stokes equations on mixed-element grids, consisting
 of triangles and quadrilaterals in 2D, and of tetrahedra, pyramids, 
 triangular prisms and hexahedra in 3D.
 Some of the findings have already been reported in
 Sayma et al. \citeyear{Luca:10}.
%
%
\paragraph{Edge-data Laplacian weight.}
%
 A novel feature of the spatial discretisation employed in
 the ALiNNS flow solver is the use of a edge-data based Laplacian weight.
 This Laplacian weight is used to compute the Laplacian terms of the
 viscous fluxes and it results in a nearest neighbour stencils.
 Chapter \ref{flow_model.chap} describes how to evaluate the Laplacian weight
 for mixed-element meshes using an approximation of the
 Galerkin finite volume {\em node-pair} formula.
 The findings are also reported in Sbardella \& Imregun \citeyear{Luca:7,Luca:11}.
%
%
\paragraph{Preconditioned Multigrid.}
%
 A preconditioned directional-implicit agglomeration multigrid method
 has been developed for the solution of the linear and non-linear
 Navier-Stokes equations on highly anisotropic 2D and 3D unstructured hybrid grids.
 The coarse grid levels are constructed
 automatically from the fine grid by agglomerating fine grid control
 volumes together. Since the coarse grid control volumes may have arbitrary
 polygonal shapes, the type of elements constituting the fine grid is
 irrelevant. The discrete equations on coarse grid levels
 are assembled automatically without the explicit creation of a coarse grid.
 The multigrid smoother consists of a preconditioned
 point- or line-Jacobi Runge-Kutta relaxation algorithm, which guarantees
 efficient damping of high-frequency error modes on highly stretched grids.
 Appendix \ref{multigrid.chap} gives a literature review of preconditioned
 multigrid methods and describes the actual multigrid algorithm that is
 implemented in ALiNNS.
%
% 
%
\subsection{Understanding of unsteady flows}
%
 Chapter \ref{rt27.chap} presents a detailed  numerical analysis of a stator-rotor
 interaction in the case of typical HP turbine stage
 using both linear and non linear unsteady flow representations.
 The main contributions are summarised below.
%
\begin{itemize}
%
\item
 Detailed analysis of the rotor steady-state flow with discussion
 of secondary and tip leakage flow.
%
\item
 Calculation of the aerodynamic forcing functions from the
 stator outlet steady-state solution. These forcing functions
 are obtained splitting the non-linear steady-state flow
 into potential and vortical components using the theory
 reported in Appendix \ref{waves.chap}.
%
\item
 Analysis of the distinct influence of potential and vortical forcing functions
 on the rotor unsteady aerodynamics using a time-linearised approach.
%
\item
 Assessment of the predicted results by comparing them with
 experimental data and predicted non-linear results.
%
\end{itemize}
%
