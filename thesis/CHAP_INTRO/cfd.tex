%
%
%
%
%
\section{CFD Methods for Compressible Flows}
\headb{Introduction}{CFD methods for compressible flows}
\label{CFD_section}
%
 Both linear and non-linear unsteady flow CFD methods
 are primarily based on the solution of compressible
 inviscid or turbulent viscous flow equations.
 CFD techniques have evolved rapidly as a discipline
 and are increasingly being used to complement wind tunnels,
 both for preliminary and advanced design.
 Based on the mathematical foundations laid by Godunov \citeyear{Godunov:1} and
 Lax \citeyear{Lax:1} among others, the field has come into its own in the
 last twenty years.
 Very significant advances have been made in the areas of spatial discretisation,
 grid generation and solution strategies. Tremendous advances in computer 
 architecture and parallel processing have also contributed to the very
 rapid advances seen over the last ten years.

 When dealing with the numerical discretisation of the Navier-Stokes equations,
 several distinct issues need to be addressed. 

 i) Spatial discretisation of the inviscid fluxes.
 The system of Euler equations mathematically
 belongs to a class called {\em hyperbolic systems of conservation laws} which
 have, as their main feature, the possibility to develop discontinuities even
 when the initial data are smooth.
 This means that a numerical algorithm needs to be designed to be able to deal with
 such possibilities.

 ii) The discretisation of the computational domain with
 structured, block-structured, unstructured or mixed-element grids.

 iii) Explicit or implicit time integration.
 This represents a very broad distinction since
 several other techniques, such as preconditioning and multigrid, can be
 employed in order to accelerate convergence. 

 iv) Boundary conditions. The stability and accuracy of the numerical scheme
 may hinge on how the boundary are treated.
%
%
%
\subsection{A brief review of flux discretisation methods}
\label{flux_discretisation.sec}
%
 The accurate and efficient discretisation of the hyperbolic conservation
 laws of fluid dynamics has been studied by several authors.
 Roe \citeyear{Roe:1} developed an approximate version of the Riemann solver
 used by Godunov \citeyear{Godunov:1} in his pioneering work,
 and this approach is still one of the best numerical schemes
 for the discretisation of the Euler equations.
 The Roe scheme belongs to the class of methods called
 {\em Flux Difference Splitting} (FDS) where
 the numerical flux function is obtained through a flow decomposition into
 acoustic, entropy and vorticity waves. Each wave is
 then discretised according to its propagation speed in an upwind
 way.
 Other FDS methods have been developed by Osher \citeyear{Osher:1}, Pandolfi
 \citeyear{Pandolfi}, Harten \citeyear{Harten:2}, Leveque \citeyear{Leveque:1}
 and Einfeldt \citeyear{Einfeldt:1,Einfeldt:2}.
 
 In parallel with the FDS methods, a different class of schemes has been developed
 where the spatial discretisation is obtained using central
 differencing stabilized by an artificial dissipation term.
 Such methods, developed Beam \& Warming \citeyear{Beam:1}
 and by Jameson et al. \citeyear{Jame:1},
 are computationally more efficient then the FDS ones but their stability
 relies on the explicit addition of artificial viscosity.
 This requires the empirical determination of the so-called
 ``free parameters''.
 On the other hand, the upwind FDS method are considered to be parameter
 free.
 A third class of methods, first developed by Stager \& Warming
 \citeyear{Stager:1} and by van Leer \citeyear{Leer:5}, is known as
 {\em Flux Vector Splitting} (FVS).
 These methods introduce, in the numerical discretisation, the signs
 of the wave propagation speeds (eigenvalues of the flux Jacobian)
 but they do not attempt to solve an approximate Riemann problem.
 Different types of FVS have been developed over the last decade
 with the goal of devising an accurate numerical scheme for capturing
 shock and contact discontinuity with minimal numerical dissipation
 and oscillation.
 Among other FVS methods, it is worth mentioning the AUSM (Advection
 Upstream Splitting Method) scheme of Liou \& Steffen \citeyear{Liou:1,Liou:2},
 the CUSP (Convective Upwind Split Pressure) scheme by
 Jameson \citeyear{Jame:2,Jame:7} and the gas kinetic analogy of
 Prendergast \& Xu \citeyear{Prendergast:1}.
 
 Several authors extended the schemes above to second order accuracy
 (labelled high resolution schemes by Harten), using different mathematical
 formulations.
 Harten \citeyear{Harten:1} introduced the important concept of TVD
 (Total Variational Diminishing) to extend Roe's first order
 flux-difference scheme to second order.
 Van Leer \citeyear{Leer:1,Leer:2,Leer:3,Leer:4}
 tackled the problem in a geometric sense introducing the MUSCL schemes (Monotone 
 Upstream-centered Schemes for Conservation Laws).
 Osher \citeyear{Osher:1}, Osher \& Chakravarthy \citeyear{Osher:2}
 contributed to the mathematical foundation of these concepts and providing
 the CFD community with solid basis to simulate high-speed flows
 with shock waves and other type of discontinuity.

 One of the most spectacular outcomes of the analysis and development
 of TVD schemes is the realization that a bridge can be established
 between the upwind FDS, FVS and central discretisation schemes.
 This allows central schemes to be formulated with adapted dissipation
 satisfying TVD requirements.
 Key papers which show the relationship between central difference and
 upwind schemes are given by Swanson \& Turkel \citeyear{Turkel:1}
 and Jorgenson \& Turkel \citeyear{Turkel:2}.
 Comparisons of different numerical schemes are given by
 Woodward \& Colella \citeyear{Colella:1} and
 Swanson et al. \citeyear{Turkel:3}.
%
%
\subsection{Structured versus unstructured grid debate}
%
 The mathematical formulation of the above schemes is developed in 1D
 because the TVD analysis is not possible for multidimensional problem.
 On structured body-fitted grids, these 1D models were extended to 2D and 3D
 in a natural by using the so-called generalized coordinates.
 However, the task of generating efficient structured grids for complex
 configurations remains a serious challenge.
 The desire of computing flows over complex configurations spawned a surge of activity in the area
 of unstructured grids\footnote{The term unstructured refers to grids
 where the number of cells surrounding a typical node does not
 necessarily remain constant.}.
 The development of both structured and unstructured grid solvers will
 now be examined in some detail.
 
 In recent years, the rapid development of numerical methods for the solution
 of the flow equations and the availability of powerful computers 
 led to the emergence of various systems for the prediction of complex
 turbomachinery flows.
 Most  prediction methods for turbomachinery flows use structured
 grids (Denton \citeyearNP{Denton:3,Denton:1}, Rai \citeyearNP{Rai:1,Rai:2},
 Dawes \citeyearNP{Dawes:1},
 Arnone \& Swanson \citeyearNP{Arnone:2}, Arnone \citeyearNP{Arnone:1},
 Saxer \& Felici \citeyearNP{Saxer:1})
 On the other hand, unstructured grids received a great deal of research
 and development for external compressible flows.
 Barth \& Jespersen \citeyear{Barth:1},
 Barth \citeyear{Barth:4,Barth:2,Barth:3},
 Peraire et al. \citeyear{Peiro:2},
 Frink \citeyear{Frink:2,Frink:1},
 Mavriplis \citeyear{Mavriplis:4},
 and Venkatakrishnan \citeyear{Venkata:1} give
 overviews of the status of compressible Euler and Navier-Stokes solvers on
 unstructured grids.
 Different spatial and temporal discretisation options
 for steady and unsteady flows are discussed.

 It is only in recent years that unstructured tetrahedral grids found
 their way into 3D turbomachinery applications
 (Dawes \citeyearNP{Dawes:2,Dawes:3}, Vahdati \& Imregun \citeyearNP{Mehdi:3},
 Sayma et al. \citeyearNP{Luca:10}).
 While unstructured grid methods  provide  flexibility for discretising
 complex geometries, they have the drawback of requiring larger
 computer storage and more CPU effort than their structured counterparts.
 Due to the relatively simple shape of turbomachinery blades, structured grids
 are considered to be as the most suitable 
 route by many researchers for their discretisation.
 Design considerations increasingly require the inclusion of complex features
 such as over tip gap leakage, cooling holes in turbine blades,
 snubbered fan blades, fan assemblies with intake ducts, struts and various
 other structural elements. Due to the complexity of such geometries,
 the natural way forward is to use unstructured grids. 
 Although tetrahedral grids, the choice of which appears obvious,
 are relatively easy to generate for inviscid flow calculations and away
 from the walls for viscous flow calculations, the situation becomes more
 complicated in boundary layers, where large aspect ratio cells are
 required for computational efficiency.
 In large parts of the solution domain, the gradients normal to the walls
 are several orders of magnitude larger than those along the walls,
 thus more grid points are required  in the former direction than in the latter.
 Tetrahedral grids are not ideal for use in boundary
 layers where very small angles degrade the accuracy of the solution
 (Aftosmis et al. \citeyearNP{Aftosmis:1}).

 Such considerations led to the development of hybrid grid models
 where hexahedral or prismatic cells can be used in the boundary layers
 and tetrahedral and prismatic cells can be used to fill the domain away
 from the walls.
 The idea of using mixed elements in an unstructured mesh technique is by no means novel,
 and has been previously advocated by several authors (Barth \citeyearNP{Barth:4},
 Parthasarathy et al. \citeyearNP{Kallinderis:2}).
 In fact, many have recognized the benefits
 of mixed elements, but have nevertheless advocated the use of standard tetrahedral 
 grids, due to the homogeneity in the data structures and relative
 discretisation simplicity.
 Mavriplis \& Venkatakrishnan \citeyear{Mavriplis:3}, 
 Moinier et al. \citeyear{Giles:9} presented unified solution techniques
 which can handle mixed-elements grids.

 For turbomachinery blades, Sbardella et al. \citeyear{Luca:3,Luca:9}
 presented a method to generate hybrid semi-structured grids  
 where the boundary
 layers are filled with hexahedral cells and the rest of the domain is filled
 with triangular prisms. Such a route not only provides a very efficient
 spatial discretisation over  standard
 unstructured grids but it also provides a much better grid quality over its  
 fully structured counterparts.
 When dealing with 3D blades, the present work will use semi-structured
 grids for their computational efficiency,
 although the solver is written for general hybrid unstructured grids.
 This is achieved
 by using an edge-data structure to construct the discretisation over all
 element types.
 In this approach, the grid is presented to the solver
 as a set of node pairs connected by edges. The edge weights representing
 the inter-cell boundaries are computed in a separate preprocessor stage.
 Consequently, the solver has a unified data structure where the  
 nature of the hybrid mesh is concealed from the main calculations loops.
%
%
\subsection{Time integration using multigrid methods}
%
 Traditionally, time integration techniques may be classified into
 five groups: explicit, implicit direct method, iterative implicit
 method, preconditioning techniques and multigrid methods.
 However, some techniques can be shown to be equivalent to others 
 and state-of-the-art time-stepping algorithms combine techniques
 from several areas.
 Venkatakrishnan \citeyear{Vankata:3} gives an overview of the
 different time marching techniques for solving both steady
 and unsteady flows.
 Unstructured grid methods are compared to those used for structured,
 body-fitted grids. The convergence characteristics of
 various implicit methods are compared for a number of test cases.
 
 Explicit methods represent a straightforward way of integrating the
 system of ordinary differential equations (ODEs) which arises from the
 spatial discretisation, since they do not require any matrix inversion.
 Multistage Runge-Kutta explicit methods
 have been widely used in the CFD community (Jameson et al. \citeyearNP{Jame:1},
 Jorgenson \& Chima \citeyearNP{Chima:1}).
 The coefficients for these Runge-Kutta schemes techniques are optimised
 in order to yield a large CFL number and good damping properties
 (Jameson \citeyearNP{Jame:4}, Van Leer \citeyearNP{Leer:7}).
 Local time stepping (Jameson et al. \citeyearNP{Jame:1}, Denton \citeyearNP{Denton:3})
 and residual smoothing
 (Jameson et al. \citeyearNP{Jame:1}, Arnone \citeyearNP{Arnone:1})
 are employed to accelerate convergence.
 Even with this methodology, the convergence to steady state is usually
 unacceptably slow. For unsteady flow, multistage Runge-Kutta algorithms
 are ideal for high-frequency unsteadiness\footnote{This is the case
 of computational aeroacoustics.} where the choice of the time-step
 is not restricted by stability considerations.
 However, the may result inefficient for low-frequency unsteady flow
 and for viscous turbulent flow simulations.
 
 Implicit direct methods, in where the Jacobian matrix is directly inverted, are
 non-competitive for large problems since the matrix inversion procedure
 scales with $n\se{3}$, where $n$ is the number of unknowns
 (Venkatakrishnan \& Barth \citeyearNP{Venkata:5}).
 Moreover the governing equations are non-linear, thus a direct method cannot produce
 an answer in a single iteration and must therefore be used iteratively.

 Rather than inverting the Jacobian matrix directly at each time-step, a
 linearised system of equations may be solved at each time-step using an iterative method.
 This can substantially reduce the overall computational requirements since the
 approximate linear system need not be solved to a high degree of accuracy.
 Also, iterative methods generally exhibit lower computational complexity than
 direct inversion methods.
 In deriving the implicit system, a common practice is the use of
 a {\em defect correction procedure} which leads to a discretised
 Jacobian matrix which is a order less accurate than
 the right-hand side (Tidriri \citeyearNP{Tidriri:1},
 Mavriplis \citeyearNP{Mavriplis:5}).
 This is not only due to storage considerations and computational complexity,
 but also to the fact that the resulting lower-order matrix is better conditioned.
 However, the mismatch between the discretisation and the Jacobian operators implies
 that the resulting system is only moderately implicit.
 For unstructured meshes, the formulation of an efficient iterative technique
 can be quite complicated. For this reason, point-Jacobi iterative procedures
 have been widely used (Brenneis \& Eberle \citeyearNP{Brenneis:1},
 Anderson and Bonhaus \citeyearNP{Anderson:1},
 Sayma et al. \citeyearNP{Luca:10}).
 Slack et al. \citeyear{Slack:1}  compare the point-Jacobi,
 Gauss-Seidel and LU decomposition time integration techniques for inviscid flows
 on unstructured meshes.

 Krylov methods\footnote{A review of Krylov methods is given by
 Venkatakrishnan (1995).} represent an alternative
 to iterative methods. The general idea
 is to obtain improved updates to the solution by using information generated at
 previous updates.
 There are a number of different Krylov methods which have been developed,
 but for CFD problems the most prevalent Krylov method is the GMRES method
 (Saad \& Schultz \citeyearNP{Saad:1}).

 These classical implicit methods work relatively well for inviscid-flow
 calculation but they may prove inadequate for high Reynolds number viscous flows
 because of two reasons: (i) the need to use high aspect
 ratio cells to represent the steep gradients in the
 boundary layer regions, and (ii) the overall increase in the number
 of mesh points for practical 3D aerodynamic configurations.
 A suitable algorithm should be devised in order to take into account the
 interaction between the discretised method, the computational mesh
 and the physics of the viscous flow.
 Moreover, the right balance between the operation count,
 storage requirements and parallel scalability is also very important.
 A general solution strategy, which is suitable for devising
 efficient solution algorithms, is the multigrid approach
 (Wesseling \citeyearNP{Wesseling:1}).
 The basic idea of a multigrid strategy
 is to accelerate the rate solution convergence on a fine grid
 by using a series of coarser grids.
 
 Explicit multigrid smoothers have been widely used in the CFD community
 for solving both steady and unsteady flows. 
 A popular explicit multigrid method is the semi-discrete scheme of Jameson
 et al. \citeyear{Jame:1} which uses multi-stage Runge-Kutta time stepping
 to integrate the system of ODEs resulting from
 the spatial discretisation. Local time-stepping was also used
 in order to accelerate convergence.
 Several researchers used the technique proposed by Jameson
 et al. \citeyear{Jame:1} to obtain efficient solution procedures:
 Jameson \citeyear{Jame:4,Jame:2},
 Denton \citeyear{Denton:3},
 Mulder \citeyear{Mulder:1,Mulder:2}
 Peraire et al. \citeyear{Peiro:3},
 Mavriplis \& Martinelli \citeyear{Mavriplis:1},
 Mavriplis \& Venkatakrishnan \citeyear{Mavriplis:8},
 Mavriplis \citeyear{Mavriplis:4},
 Melson et al. \citeyear{Melson:1},
 Parthasarathy et al. \citeyear{Kallinderis:2} and
 Arnone \citeyear{Arnone:1}.

 However, despite considerable success for Euler computations,
 the multigrid approach have shortcomings for Navier-Stokes computations,
 especially high aspect ratio cells are used inside the boundary layer
 region.
 One way of decreasing the discrete stiffness caused by highly stretched cells
 is to use of a matrix time step or preconditioner.
 Recently, considerable amount of research has been devoted to the formulation
 of preconditioner multigrid algorithms for computing high Reynolds number
 viscous flows.
 Point-Jacobi preconditioners (Pierce \& Giles \citeyearNP{Giles:10},
 Pierce et al. \citeyearNP{Giles:11}) and line-Jacobi preconditioners
 (Mavriplis \citeyearNP{Mavriplis:6,Mavriplis:7}) have been reported in the
 literature.
 While point-Jacobi and line-Jacobi preconditioners are used to
 decrease the stiffness arising from the grid anisotropy in boundary layer regions,
 another class of preconditioners is also available for reducing the stiffness
 associated with low-speed compressible flows:
 Turkel \citeyear{Turkel:4,Turkel:5}, Turkel et al. \citeyear{Turkel:7},
 Van Leer et al. \citeyear{Leer:6}, Choi \& Merkle \citeyear{Choi:1}.
 An overview of modern acceleration techniques, such as preconditioned multigrid,
 is given by Mavriplis \citeyear{Mavriplis:5}.

 Appendix \ref{multigrid.chap} presents a detailed literature review of the
 current state-of-the-art multigrid techniques and develops a hybrid-grids
 Jacobi-preconditioned multigrid method for turbomachinery
 steady and unsteady flow computations.
%
%
%
%
\subsection{Non-reflecting boundary conditions}
%
 For convection dominated phenomena, such as those described by the 
 compressible Navier-Stokes equations, the formulation of correct boundary
 conditions is extremely important and its impact on the numerical scheme
 is often dominating.
 The reason for this strong influence can be traced back to the physical nature of
 the convection propagation phenomena (Hirsh \citeyearNP{Hirsch:1}).
 When obtaining a numerical solution for the Euler or Navier-Stokes
 equations, one has no choice but to truncate the computational domain.
 This is particularly true for internal flow simulations where
 the inlet/outlet boundaries are placed typically less
 than one chord away from the blade.
 For such configurations, the far-field flow contains a significant
 component of several different wave numbers, especially
 for flows which are supersonic in the flow direction but
 subsonic in the axial direction. In this case shocks propagate indefinitely
 and can be reflected by improper boundary conditions.

 The term non-reflecting boundary condition (or absorbing boundary conditions)
 indicates a far-field boundary treatment which should prevent any
 non-physical reflection of waves which are leaving
 the computational domain.
 Specialized treatments exist for steady-state flows, unsteady
 frequency-domain flows and unsteady time-domain flows.
 The milestone paper by Engquist \& Majda \citeyear{Engquist:1} reported
 a hierarchy of approximate non-reflecting boundary conditions for
 multidimensional problems. The authors constructed a non-local
 perfectly absorbing boundary condition for the scalar wave equation
 and derived highly absorbing local approximations by mean of Taylor 
 series around the incidence angle of the outgoing waves.
 The first-order approximation represents the so-called {\em 1D
 approximation}, or {\em method of characteristics},
 where only the waves which travel in the direction normal to the
 far-field boundary are perfectly absorbed.
 This approach is the most commonly used one for time-domain
 unsteady flow computations (Thompson \citeyearNP{Thompson:1,Thompson:2}).
 
 An overview of different types of non-reflecting boundary conditions
 is given by Givoli \citeyear{Givoli:1} for a large number of problems.

 Unfortunately, the paper by Engquist \& Majda \citeyear{Engquist:1}
 was written for mathematicians,
 specialized in the analysis of partial differential equations and hence
 its acceptance and implementation by the CFD community has been slow.
 Hedstrom \citeyear{Hedstrom:1} applied the 1D
 boundary conditions of Engquist \& Majda to the unsteady 1D
 Euler equations.
 Higdon \citeyear{Higdon:1} discussed the initial-boundary value problem
 theory for linear hyperbolic systems and gave a physical interpretation
 in terms of wave propagation. Furthermore, Higdon \citeyear{Higdon:2,Higdon:3} 
 formulated a discrete numerical version of the analytical absorbing boundary
 condition of Engquist \& Majda \citeyear{Engquist:1}.

 During the late 1980s, a considerable amount of manpower was devoted towards
 the formulation and implementation of accurate boundary conditions for the
 multidimensional Euler equations.
 Engquist \& Gustafsson \citeyear{Engquist:2} investigated steady-state
 non-reflecting boundary conditions and their impact on the rate of
 convergence in practical steady-state flow calculations.
 Gustafsson \citeyear{Gustafsson:1} applied Engquist \& Majda \citeyear{Engquist:1}
 theory to the 2D time-dependent Euler equations.
 For such a problem, the perfectly absorbing boundary condition
 requires first a Laplace-transform at the boundary, and then
 a Fourier-transform along in the boundary.
 Such exact non-local boundary conditions are, in a mathematical
 sense, the best possible solution of the problem. However,
 they are computationally cumbersome and they are only
 available for simple problems with constant properties.
 
 Giles \citeyear{Giles:5,Giles:6} reported a unified theory for
 the construction of steady-state and unsteady non-reflecting
 boundary conditions for the Euler equations. The analysis
 was carried out using a decomposition of the linearised Euler equations
 into Fourier modes.
 The single-frequency boundary conditions were used for
 constructing both the steady-state version (no time
 variation at steady-state, thus frequency equal to zero)
 and the unsteady frequency-domain version (single known frequency
 assumed for the unsteadiness).
 For non-linear unsteady aerodynamics, Giles constructed approximate versions
 based on Taylor expansions, similar to the formulation by
 Engquist \& Majda \citeyear{Engquist:1}.
 Giles \citeyear{Giles:6} applied the steady-state version of these boundary
 conditions to 2D turbomachinery flows, showing their effectiveness in
 avoiding numerical reflections from the computational inflow/outflow
 boundaries.
 An extension to 3D is reported by Saxer \& Giles \citeyear{Giles:7}.
 Ferm \citeyear{Ferm:1} presented accurate steady-state boundary-conditions
 for channel-flow configurations. He also studied the importance
 of a correct implementation in order to enhance convergence.

 Although accurate boundary condition formulations are now available
 to the CFD community, their application has been mainly restricted
 to steady-state flows or linearised frequency-domain unsteady flows.
 The implementation of accurate non-reflecting boundary conditions
 for time-domain unsteady flows has been reported in
 computational aeroacoustics (Tam \& Webb \citeyearNP{Tam:1}).
 For these applications, the boundary conditions are constructed from
 an asymptotic solution of the governing equations for large
 distances, the so-called {\em radiation boundary conditions}.
 A comparison between 1D characteristic
 (Thompson \citeyearNP{Thompson:1,Thompson:2}), Fourier
 decomposition (Giles \citeyearNP{Giles:5,Giles:6}) and
 radiation boundary conditions (Tam \& Webb \citeyearNP{Tam:1}),
 is given by Hixon et al. \citeyear{Hixon:1},
 in the framework of computational aeroacoustics.
 It is shown that, for unsteady time-domain 
 method, the only acceptable outflow boundary treatment is that by 
 Tam \& Webb \citeyear{Tam:1}. Other schemes become
 acceptable only in special cases when the flow is nearly
 perpendicular to the boundary.
 However, the radiation boundary conditions by Tam \& Webb \citeyear{Tam:1}
 were developed for the case of a uniform mean flow only.

 An interesting way forward could involve the use of
 the absorbing boundary conditions for linearised Euler equations
 via a perfectly matched layer (PML). This method, developed by
 Hu \citeyear{Hu:1}, uses additional computational regions adjacent to the
 far-field boundaries of the computational domain. In these regions
 the governing equations are modified so that outgoing waves are
 either damped, accelerated to supersonic conditions, decelerated
 or attenuated by combinations thereof.
 Analysis and applications of such sophisticated boundary conditions
 can be found in Tam et al. \citeyear{Tam:1}, Hesthaven \citeyear{Hesthaven:1}
 and Abarbanel et al. \citeyear{Hesthaven:2}.
