%
%
%
%
%
\section{Introduction}
\label{intro_nonlinear.sec}
\headb{Nonlinear Navier-Stokes solver}{Introduction}
%
 Over the last decade, significant advances have been made
 in the area of turbulent-viscous simulations using unstructured
 grids.
 However, compared with their structured counterparts,
 standard unstructured grid solvers have lower computational
 efficiency in terms of speed and storage.
 Unstructured grids often use tetrahedral elements only,
 an approach which often leads to numerical problems when the region to be
 discretised has a preferred direction such as the boundary layer for a
 high-Reynolds number flow. However, there are no fundamental difficulties in 
 extending tetrahedral meshes to include further element types such as triangular prisms, 
 pentahedra and hexahedra.  
 Although both the discretisation of the computational domain and the flow solver
 will become more complex,  such a mixed-element approach will offer
 a better, more efficient approximation than using tetrahedral elements only.
 For instance, hexahedral elements will handle boundary layer flows much better
 than tetrahedral elements because they can be made very slender without
 creating excessively small internal angles. 
 In order to handle mixed-element meshes, the spatial discretisation
 of the governing equations needs to be formulated in such a way
 that the numerical algorithms can be applied in a uniform way to all
 element types.
 A relatively simple way of achieving such consistent numerical treatment
 is to employ an edge-based data structure, which can be obtained
 from either a finite volume (FV) or a finite element (FE) formulation
 if the mesh consists of tetrehedral elements only. In this particular case, 
 all nodes are connected directly without any internal diagonals.
 However, quadrilaterals in 2D and hexahedrals in 3D have not only
 edges but also diagonal links. Therefore, to create an edge-based
 data structure from mixed meshes, one cannot use a FE technique
 because of the non-zero shape function contributions from the
 diagonally-opposed nodes for which there are no direct edges.
 Consequently, we will use a FV technique to obtain an edge-based
 data structure from mixed element meshes but, as reported by
 Barth \citeyear{Barth:4}, Parthasarathy et al. \citeyear{Kallinderis:2},
 Mavriplis \& Venkatakrishnan \citeyear{Mavriplis:3}
 for unstructured meshes of tetrahedra, 
 the discretisation of the viscous terms, remains a major problem. 
 The FV edge-based data structure is usually obtained by
 discretising the viscous terms in two sequential  loops, the so-called
 two-loop approach. The first loop is used to construct the gradients
 at all points, and the second one forms the second derivatives from
 the computed gradient information.
 Using such a technique, 
 the viscous fluxes are treated in an analogous manner to
 the inviscid ones and no extra storage is required.
 However, this strategy, used by several authors
 (see Peraire et al. \citeyearNP{Peiro:2}, Vahdati \& Imregun \citeyearNP{Mehdi:3})
 has at least three serious drawbacks.
 First, the {\em odd-even decoupling} can destabilize
 the numerical scheme in regions where the viscous effects are
 important, e.g. the boundary layer.
 Second, for a 1D mesh of spacing $h$, the scheme will reduce to a second
 difference on a stencil of $2h$, a feature which will lower numerical
 accuracy.  Since packing enough points into the viscous layer
 is one of the main difficulties associated with viscous flow
 computations, a scheme that operates on every other point is highly
 undesirable.
 The third problem is the difficulty of implementing a viscous Jacobian which
 becomes important if an implicit time integration scheme is employed.

 An alternative approach, based on Galerkin
 finite element (GFE) approximation where velocity
 and temperature are made dependent variables,  is the derivation of the six node-pair
 coefficients for the Hessian matrix (Mavriplis \citeyearNP{Mavriplis:4},
 Selmin \& Formaggia \citeyearNP{Formaggia}).
 Each coefficient  is associated with  two nodes only, hence the term "node-pair GFE". 
 As mentioned earlier, such a formulation is not edge-based for non-tetrehedral meshes because 
 of the possible diagonal links between the nodes of a hexahedral element.
 On the other hand, the final discrete viscous
 terms of such a GFE formulation form a nearest neighbour stencil.
 Therefore, using the  Hessian node-pair
 coefficients, six second derivatives can be calculated for each node.
 Since the viscous terms can be expressed as a summation of the product
 of the node-pair coefficients and the unknowns,
 the construction of a viscous Jacobian becomes straightforward by explicit
 differentiation. Furthermore, the odd-even decoupling is avoided and the accuracy is improved.
 Unfortunately, this approach needs additional storage for the six node-pair
 coefficients and its applicability is restricted to triangular/tetrahedral
 elements if an edge-based data structure needs be employed.

 In summary, when dealing with mixed-element meshes, FV formulations can
 yield edge-based data structures but viscous flux discretisation
 remains problematic. On the other hand, GFE formulations are more
 efficient but the edge-based data structure cannot be preserved for non tetrahedral elements.  
 Therefore, the discretisation of the viscous fluxes
 can be improved by combining  the
 storage efficiency of the two-loop FV approach  with the
 numerical efficiency of the node-pair GFE method in an edge-based FV framework.
