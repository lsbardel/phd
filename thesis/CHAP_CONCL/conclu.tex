%
%
%
%
\chapter{Conclusions and Raccomendations for Further Work}
\label{conclusion.chap}
\heada{Conclusions}
%
 A unified flow solver for the solution of the steady-state, non-linear
 time-marching and linearised unsteady Favre averaged Navier-Stokes equations
 on 2D and 3D turbomachinery passages has been presented,
 analysed and implemented.
 The spatial discretisation has been obtained using a finite volume scheme
 on mixed element meshes, consisting of triangles and quadrilaterals in 2D,
 and of tetrahedra, pyramids, triangular prisms and hexahedra in 3D.
 The resulting finite volume method is a node centered edge-based algorithm
 which is applied in a uniform way to all element types.
 An important feature of the method is the derivation of an edge-based coefficient
 for the discretisation of the Laplacian operator, which results in a nearest
 neighbour stencils.
 The time relaxation for both the steady state and the linearised unsteady
 version of the code has been obtained using the same preconditioned
 Runge Kutta time stepping algorithm. This multistage time-integration technique
 has been used as the smoother for a full approximation storage
 agglomeration multigrid method.
 For non-linear unsteady aerodynamics a dual time stepping technique
 has been implemented. The scheme is fully implicit and uses the same preconditioned
 multigrid algorithm developed for steady-state predictions, in order to iteratively
 invert the equations at each physical time step.
 The numerical procedure has been tested and compared with available analytical and
 experimental data in order to assess the validity, the accuracy and the efficiency
 of the method.

 The three different numerical flow representations have been used for
 a detailed computational analysis of a stator-rotor interaction in a typical
 high pressure turbine stage. Steady state nd unsteady results have been
 compared with the experimental data obtained at Osney Laboratory.
 Also comparison between linear and non-linear unsteady results has been
 reported in order to assess the range of applicability of the linear
 method.
%
%
%
\paragraph{Highlights of the numerical procedure}

~\newline
~\newline
%
(i)
 The semi-structured mesh generator presented in Chapter \ref{mesh.chap}
 combines together the advantages of both structured and unstructured grids.
 Particular feature of the blade passage, such as leading- and trailing-edges,
 wake region are discretized without generating superfluous points.
 Moreover complex geometric feature such as tip-gap are handled in a straightforward
 way.

~\newline
 (ii)
 The use of hybrid grids improves standard unstructured schemes in terms of
 accuracy, speed and storage. Accuracy is improved by using quadrilateral/hexahedral
 elements in regions where highly stretched cells are necessessary to resolve
 steep directional gradients (i.e. boundary layer regions). In these regions, standard
 triangular/tetrahedral elements would produce angles which are virtually zero.
 In such situations a finite-volume discretisation may suffer an accuracy degradation
 due to the irregular shape of the resulting median-dual control volume.
 Another advantage of hybrid grids, over standard unstructured grids,
 is a decreased cost of the discretisation itself. For a given distribution of
 vertices, a tetrahedral mesh discretisation is roughly twice as expensive to
 evaluate as a hexahedral mesh discretisation. A decreased number of edges in the mesh
 is the reason for this difference.
 
~\newline
 (iii)
 A further advantage of the hybrid grid solver presented is its
 flexibility in handling structured grids, block structured grids, unstructured
 grids or combination of thereof without any modifications.

~\newline
 (iv)
 The viscous fluxes are split into two components: Laplacian terms and
 mixed derivative terms. The former terms contain Laplacian operators
 only and they are discretized using a side-data Laplacian weight which
 results in a nearest neighbour stencils. The mixed derivative terms
 are evaluated from the gradient informations and thus their discretisation
 does not result in a compact stencil.
 The compact stencil of the Laplacian terms is of paramount importance in
 the boundary layer region, avoiding even-odd decoupling and allowing
 a stronger diagonal dominance in the preconditioner matrix employed in the
 relaxation algorithm.

~\newline
 (v)
 The time relaxation algorithm uses an agglomeration multigrid method.
 The coarse grid levels are constructed
 automatically from the fine grid by agglomerating fine grid control
 volumes together. Since the coarse grid control volumes may have arbitrary
 polygonal shapes, the type of elements constituting the fine grid is
 irrelevant. In this way the discrete equations on coarse grid levels
 are assembled automatically without the explicit creation of a coarse grid.

~\newline
 (vi)
 The multigrid smoother consists of a preconditioned
 point- or line-Jacobi Runge Kutta relaxation algorithm, which guarantees
 efficient damping of high frequency error modes on highly stretched grids.
 The line-Jacobi preconditioner is used for viscous flow calculation
 which use a no-slip conditions at solid walls.
 The point-Jacobi preconditioner is used for calculation which
 use a flow-tangency condition at solid walls.
%
%
%
%
\paragraph{Importance of linearised unsteady viscous representations}

~\newline
~\newline
 (i)
 A linearised unsteady Navier-Stokes representations 
 offers the possibility of predicting unsteady flows at
 off-design conditions where viscous
 effects are not limited to the boundary layer.
 
~\newline
 (ii) Steady-state viscous effects are often important even at design conditions.
 In fact, Navier-Stokes simulations are used in order to predict secondary flow
 effects, correct mass flow in compressor passages where the adverse pressure gradient
 cause a thicker boundary layer, tip-leakage flow, losses caused by the trailing-edge
 base flow in turbine blades, etc.
 Using a linearised unsteady Navier-Stokes representations means
 that the same computational mesh can be used for both calculating the steady
 base flow and the unsteady perturbations.
 Interpolation of the steady-state unknown from viscous grids into inviscid ones
 are avoided and no additional numerical smoothing is needed.
 
~\newline
 (iii)
 The linearisation of the turbulence model may play an important role in
 situation were viscous effect are not confined to the boundary layer
 region. When there are not such effects, the frozen turbulence approach
 should produce very similar results.
 
~\newline
 (iv)
 When a steady viscous flow calculation, which uses the law of the wall in order
 to compute the wall shear stresses, is performed,
 the issue a linearisation of the wall function arises.
 Such a linearisation has been attempted during the research project.
 However, the results obtained using a linearised wall function and
 the ones obtained neglecting unsteady wall shear stresses were
 similar. Moreover the linearisation of the wall function
 required the use of the linearised turbulence mode since
 wall shear stresses and eddy viscosity are related in the Spalart Allmaras
 model.
 For this reason the unsteady wall shear stresses were neglected
 with no need of a linearised law of the wall.
%
%
%
%
\paragraph{Applicability of linear methods for forced response prediction}
 
~\newline
~\newline
 To establish boundaries, which effectively define
 when linear methods should produce accurate results, is not a simple task.
 This is mainly  due to the large number of parameters which influence the unsteady
 flows caused by relative blade motion.
 Comparisons between linear and non-linear results should be made for
 all type of flow regimes as well as different size of stator-rotor axial spacing
 and different stator-to-rotor pitch ratios. Also,
 3D effects need to be explored in more details.
 
 From the results obtained in Chapter \ref{rt27.chap}, the following conclusion
 are drawn.
 
~\newline
 (ii) The wake-rotor interaction is strong in the crown of the blade. In this region,
 the computed results show that non-linearities are negligible.

~\newline
 (i) High unsteady pressure fluctuations, mainly caused by the upstream potential flow field,
 are located at the leading-edge of the blade. In this region non-linearities may
 play an important role even though the linearised results follow a correct trend.
 Furthermore, such discrepancies are more evident for the blade passing frequency then
 higher mode number. For this reason, the linear method should still produce
 meaningful results for interactions in where shock waves are moving in the region
 between the stator and rotor bladerows\footnote{The presence of shock waves increase
 the amplitude of Fourier modes with higher frequencies then the blade passing frequency.}.

~\newline
 (iii) Localised non-linearities are located in region where the steady-state flow
 differs from the time-average flow. This is often the case when non-linear features,
 such as shock waves, are present.

~\newline
 (iv) High unsteady pressure fluctuations are placed across the suction side legs
 of the horseshoe vortex at both hub and tip end walls. These suction side legs,
 of the horseshoe vortex,
 travel from the end walls towards mid section because of the rotation of the
 passage vortecies thus unsteadiness present a pronounce 3D behaviour.
%
%
%
%
%
\paragraph{Further Work}

~\newline
~\newline
 (i) The preconditioned agglomeration multigrid needs to be improved in order to
 enhance both performance and robustness. A more accurate prolongation operator
 need to be developed. This task is non-trivial because of the absence
 of an underlying grid. The preconditioner may cause some instabilities
 at stagnation points, where large variations of the flow angle appear
 at significant velocity values. This loss of robustness can be attributed to
 a flow-angle sensitivity thus improvement in this direction should be made.

~\newline
 (ii) Further testing of the linearised unsteady method needs to be undertaken for
 real 3D geometries. In addition the inclusion of the quadratic source
 terms (QST) (Giles \citeyearNP{Giles:13}) in the linearised frequency-domain equations
 should be implemented.
 The magnitude of such terms is a quadratic function of the level of unsteadiness.
 In this way after a first linearised calculation, the unsteadiness are used to evaluate
 the QST which are then used as source terms for a second linearised analysis.
 Such method should take into account for the differences between the steady-state flow
 and the time-average flow.
%
