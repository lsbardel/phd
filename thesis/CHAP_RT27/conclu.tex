%
%
%
%
\section{Concluding Remarks}
\headb{Transonic Turbine Stage}{Concluding remarks}
\label{rt27_conclusions.sec}
%
 An extended analysis of the steady and unsteady aerodynamics
 in a rotor passage typical of high-pressure turbine stage
 of contemporary turbomachines has been presented.
 The main highlight of this study are summarised as follow:
%
\begin{itemize}
%
\item
 A steady state 3D calculation for a coupled NGV-rotor
 configuration is essential in order to correctly predict the rotor
 primary and secondary flow as well as to evaluate the flow non-uniformities
 at the NGV outlet. A good representations of such non-uniformities is
 compulsory for the evaluation of the aerodinamic forcing functions which
 need to be imposed at the rotor inlet in a linear unsteady calculation.
 For this reson, the use of steady-state non-reflecting boundary conditions
 at the NGV outlet/rotor inlet boundaries is essential.
%
\item
 Both vortical and potential interactions are essential for a complete
 unsteady flow representation. The vortical interaction is strong
 in the crown of the blade while the potential one in the leading-edge
 region of the blade.
%
\item
 The potential-flow interactions reveals 3D effects
 which can be justified by the coupling of vortical-acoustic modes
 due to the presence of centrifugal and Coriolis forces in the
 rotor passage. This coupling is more evident in the second
 Fourier mode.
%
\item
 Non linear effect are minimal for the first two Fourier components thus
 justifying a liner unsteady approach.
%
\item
 The linear unsteady approach offeres great advantages over its non-liner
 counterpart: (i) much more efficient in term of CPU speed and storage.
 (ii) Much faster pre-processing since the linear calculations are
 performed using a single passage mesh while non-linear methods
 require the assembling of multiblade-multirow meshes\footnote{This advantage is
 even more enfatised when multigrid-solver are used}.
 (iii) Much faster post-processing analysis since the computed data
 represents the complex amplitude of the unsteady perturbation for a given
 Fourier mode.
\end{itemize}
%
 
